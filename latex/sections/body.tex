% Introduction paragraph
The nature of words is not a fixed one, but rather, words are capable of great motive power. This is one of the most important claims made in the select quotation from Sigmund Freud's \emph{Introductory Lectures of Psychoanalysis}, that words can move emotions, knowledge, and judgement. And looking back at my own experience with words, it is not hard to see why Freud exclaims that "words were originally magic".

\noindent
For what else could they be? As a child growing up in the trilingual household of the Mandarin Diaspora, I had the advantage of living with not one type of word, but with the words of three different societies. I learned early on that some types words belong in only certain places: Mandarin in the household, English at school. And as I got older, the power of the right words in the right places became even more apparent to me, especially as I honed my understanding of language in the whetstones of discussion and debate.

% Section on the emotional power of Words
Take the role of words in the conveyance of emotion, for example. \emph{"By words one person can make another blissfully happy"}, writes Freud, \emph{"or drive them to despair"}. The ability of language to induce states of deep emotion is a well known to us, but we seldom stop to consider just how \emph{bizarre} and \emph{extraordinary} it is, for one's emotional state to be driven by what appears to be ephemeral. Not only can emotions be induced through the use of words (e.g. \emph{"I love you"}), but empathy can be established, by using words as a medium of exchange. Where words are not mere drugs that induce a given state, but they allow different people to \emph{understand} the griefs of others. As if somehow, one can step into the hearts of another, just by receiving the right words.

% Section on how words convey knowledge
Hence, it is not merely just a figure of speech to say that words are \emph{moving}, but words can literally appear to transport us in some way. Indeed, this \emph{motive power} is alluded to in Freud's next claim, that \emph{"by words the teacher \emph{conveys} his knowledge"}. For when we read Harvey, Galileo, Euclid, or Roux, what happens in that process of reading, if we are not conveyed into the mind and thoughts of another? Our world, a place used to literacy --- seems to ignore the fantastical nature of such transport. Somehow, through the perception of soot marks on wood pulp, the thoughts and experiences of our authors come to life in our minds. Depending on the way such words are arranged, we may even see the world in their time, through their eyes and senses, even if our authors are dead for thousands of years.

% Section on how words can affect judgement
Mistake not, this power --- for it effectively is a form of immortality. And nowhere is it more apparent, than with the type of words that deal with reason. It is trivial to say that words can be the vehicles for reason, that judgement can be swayed through the application of words. But not only do words convey reason, certain words can become rationales in their own right, ends to themselves. We call these words by different names, such as "ideology", "dogma", "doctrine", or "creed". And this words like these are not merely immortal, but they undergo a form of apotheosis, where they are no longer controlled by their original authors, but lead lives of their own.

% Conclusion
Ultimately, words are magical for they are everywhere. And perhaps it is through their sheer \emph{omnipresence}, that we are blinded to their magical in the first place. For words were originally magic, and I think --- they still are, today.
